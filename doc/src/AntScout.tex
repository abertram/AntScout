\documentclass[a4paper,10pt]{scrreprt}
\usepackage[utf8]{inputenc}

\title{AntScout-Dokumentation}
\author{Alexander Bertram}

\begin{document}

\maketitle

\chapter{Typografische Konventionen}
\label{chap:typografische-konventionen}

Konsolen-Kommandos, URLs oder Quellcode-Ausschnitte im Fließtext werden in einer \texttt{Nicht-Proportionalschrift} dargestellt.

\chapter{Allgemeine Problemstellung}
\label{chap:allgemeine-problemstellung}

\chapter{Benutzerhandbuch}
\label{chap:benutzerhandbuch}

\section{Ablaufbedingungen}
\label{sec:ablaufbedingungen}

\section{Installation}
\label{sec:installation}

\textsc{AntScout} benötigt keine Installation.
Das Archiv muss lediglich auf die lokale Festplatte extrahiert werden.
Dabei wird der Ordner ``AntScout'' erstellt, der alles beinhaltet, was für den Start der Anwendung nötig ist.

\section{Programmstart}
\label{sec:programmstart}

\subsection{Voraussetzungen}
\label{sec:voraussetzungen}

\begin{itemize}
  \item Eine bestehende Internet-Verbindung.
  \item Eine möglichst aktuelle Java-Version von Oracle.
  \item Der lokale Port \texttt{8080} darf nicht durch eine bereits laufende Anwendung blockiert sein.
\end{itemize}

\subsection{Start}
\label{sec:programmstart-start}

\begin{enumerate}
  \item Konsole (auch Eingabeaufforderung oder Kommandozeile genannt) öffnen
  \item In das erstellte Verzeichnis ``AntScout'' wechseln
  \item \texttt{sbt}
  \item \texttt{container:start}
  \item \texttt{http://localhost:8080} oder \texttt{http://127.0.0.1:8080} im Browser aufrufen
\end{enumerate}

\subsection{Hinweise}
\label{sec:programmstart-hinweise}

\begin{itemize}
  \item Nach dem Start von \texttt{sbt} werden alle benötigten Bibliotheken heruntergeladen.
    Je nach Internet-Verbindung kann dieser Vorgang mehrere Minuten dauern.
  \item Nach der Eingabe von \texttt{container:start} werden die benötigten Karten heruntergeladen und vorverarbeitet.
    Dieser Vorgang kann auch je nach Internet-Verbindung und Computer-Leistung mehrere Minuten dauern.
  \item Es sollte nach Möglichkeit ein möglichst moderner Browser, der HTML5 unterstützt, verwendet werden.
\end{itemize}

\section{Programmende}
\label{sec:programmende}

\begin{enumerate}
  \item \texttt{container:stop}
  \item \texttt{exit}
\end{enumerate}

oder einfach die Konsole schliessen.

\section{Bedienungsanleitung}
\label{sec:bedienungsanleitung}

\section{Fehlermeldungen}
\label{sec:fehlermeldungen}

\section{Wiederanlaufbedingungen}
\label{sec:wiederanlaufbedingungen}

\chapter{Programmierhandbuch}
\label{chap:Programmierhandbuch}

\section{Entwicklungskonfiguration}
\label{sec:entwicklungskonfiguration}

\section{Problemanalyse und Realisation}
\label{sec:problemanalyse-und-realisation}

\section{Beschreibung grundlegender Datenstrukturen}
\label{sec:beschreibung-grundlegender-datenstrukturen}

\section{Programmorganisation}
\label{sec:programmorganisation}

\section{Programmtest}
\label{sec:programmtest}

\end{document}
