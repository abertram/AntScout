\documentclass[
  BCOR = 10mm,
  bibliography = totoc,
  listof = totoc,
  parskip = half
]{scrreprt}

\usepackage[
  printonlyused,
  withpage
]{acronym}
\usepackage[ngerman]{babel}
\usepackage{booktabs}
\usepackage[T1]{fontenc}
\usepackage{graphicx}
\usepackage{ifpdf}
\usepackage[utf8]{inputenc}
\usepackage{listings}
\usepackage{lmodern}
\usepackage{scrhack}
\usepackage[shadow]{todonotes}
\usepackage{tabularx}
\usepackage{hyperref}

\author{Alexander Bertram}

\bibliographystyle{alphadin}

\ifpdf
  \definecolor{darkblue}{rgb}{0, 0, 0.5}
  \hypersetup{
%     citecolor = darkblue,
    colorlinks = true,
    breaklinks = true,
    linkcolor = darkblue,
    linktoc = all,
%     menucolor = darkblue,
%     urlcolor = darkblue
%     pdfauthor = {\theAuthor},
%     pdfsubject = {\theSubject},
%     pdftitle = {\theTitle~-~\theSubtitle}
  }
\fi

% Listings-Konfiguration
\lstset{
  aboveskip = \floatsep,
  captionpos = b,
  basicstyle = \ttfamily\small,
  breaklines = true,
  frame = lines,
  numbers = left,
  numberstyle = \tiny,
  showstringspaces = false
}

\title{AntScout-Dokumentation}

\begin{document}

\pagenumbering{Roman}

\maketitle

% Verzeichnisse
\tableofcontents
\listoffigures
\listoftables
\renewcommand{\lstlistlistingname}{Listingsverzeichnis}
\lstlistoflistings
\newacronym{osm}{OSM}{\textit{OpenStreetMap}}
\newacronym{url}{URL}{Uniform Resource Locator}


\clearpage

\pagenumbering{arabic}

\chapter{Typografische Konventionen}
\label{chap:typografische-konventionen}

Konsolenkommandos, URLs oder Quellcodeausschnitte im Fließtext werden in einer \texttt{Nicht-Proportionalschrift} dargestellt.

\chapter{Allgemeine Problemstellung}
\label{chap:allgemeine-problemstellung}

Im Rahmen des Masterprojekts und der -thesis soll eine Anwendung erstellt werden, in der der Ameisenaalgorithmus AntNet auf realen Straßenkarten für dynamisches Routing implementiert wird.
Die Daten werden von \ac{osm} abgerufen und in einen Graphen umgewandelt, auf dem AntNet operieren kann.
Die Erstellung der Anwendung ist in zwei Schritte unterteilt:

\begin{enumerate}
  \item Abruf und Aufbereitung der Daten sowie die Bereitstellung des Softwarewerkzeugs zur Parametrisierung der Problemstellung
  \item AntNet-Implementierung und Einstellung der Parameter
\end{enumerate}

Der erste Schritt ist das Ziel des Masterprojekts und der zweite das Ziel der Masterthesis.

\chapter{Benutzerhandbuch}
\label{chap:benutzerhandbuch}

Im Benutzerhandbuch wird die bereits fertige Anwendung beschrieben.

% \section{Ablaufbedingungen}
% \label{sec:ablaufbedingungen}

\section{Installation}
\label{sec:installation}

AntScout benötigt keine Installation.
Falls die Anwendung als Archiv ausgeliefert wurde, muss es lediglich auf die lokale Festplatte extrahiert werden.
Dabei wird der Ordner ``AntScout'' erstellt, der alles beinhaltet, was für den Start der Anwendung nötig ist.
Falls die Applikation dagegen nicht in einem koprimierten Zustand und bereits in einen Ordner namens ``AntScout'' extrahiert ist, muss dieser Ordner auf die lokale Festplatte kopiert werden.

\section{Programmstart}
\label{sec:programmstart}

\subsection{Voraussetzungen}
\label{sec:voraussetzungen}

\begin{itemize}
  \item Eine bestehende Internetverbindung.
  \item Eine möglichst aktuelle Java-Version von Oracle.
  \item Der lokale Port \texttt{8080} darf nicht durch eine bereits laufende Anwendung blockiert sein.
\end{itemize}

\subsection{Start}
\label{sec:start}

\begin{enumerate}
  \item Konsole\footnote{auch Eingabeaufforderung oder Kommandozeile genannt} öffnen
  \item In das erstellte bzw. kopierte Verzeichnis ``AntScout'' wechseln
  \item \texttt{sbt}
  \item \texttt{container:start}
  \item \texttt{http://localhost:8080} oder \texttt{http://127.0.0.1:8080} im Browser aufrufen
\end{enumerate}

\subsection{Hinweise}
\label{sec:hinweise}

\begin{itemize}
  \item Nach dem Start von \ac{sbt} werden alle benötigten Bibliotheken heruntergeladen.
    Je nach Internetverbindung kann dieser Vorgang mehrere Minuten dauern.
  \item Nach der Eingabe von \texttt{container:start} werden die benötigten Karten heruntergeladen und vorverarbeitet.
    Dieser Vorgang kann auch je nach Internetverbindung und Computerleistung mehrere Minuten dauern.
  \item Es sollte nach Möglichkeit ein möglichst moderner Browser, der HTML5 unterstützt, verwendet werden.
\end{itemize}

\section{Programmende}
\label{sec:programmende}

\begin{enumerate}
  \item \texttt{container:stop}
  \item \texttt{exit}
\end{enumerate}

Oder die Konsole schliessen.

\section{Bedienungsanleitung}
\label{sec:bedienungsanleitung}

\subsection{Front-End}
\label{sec:front-end}

Nach dem Aufruf \texttt{http://localhost:8080} im Browser öffnet sich die Startseite.
Ein Screenshot dieser Seite ist in der Abbildung \ref{fig:startseite} zu sehen.

\begin{figure}[htbp]
  \centering
  \includegraphics[width=\textwidth]{Bilder/Startseite.png}
  \caption{Startseite}
  \label{fig:startseite}
\end{figure}

Die Seite ist in drei Bereiche unterteilt:

\begin{itemize}
  \item Menü
  \item Karte
  \item Informationen
\end{itemize}

\subsubsection{Menü}
\label{sec:menue}

Die obere Leiste stellt das Menü dar.
Hier kann zwischen der Karten- und der Monito"-ring-Ansicht\footnote{Siehe Abschnitt \ref{sec:monitoring}} umgeschaltet werden.

\subsubsection{Karte}
\label{sec:karte}

Die Karte nimmt den größten Platz der Seite ein.
Hier sind die Kreuzungen - nachfolgend Knoten genannt - zu sehen, zwischen denen navigiert werden kann und die Wege, die die Knoten verbinden.
Jeder Knoten und jeder Weg kann durch einen Mausklick markiert werden.
Dieser wird dann optisch hervorgehoben.

\subsubsection{Informationen}
\label{sec:informationen}

Der Informationenbereich auf der rechten Seite ist in drei weitere Bereiche unterteilt:

\begin{itemize}
  \item Pfadinformationen
  \item Knoteninformationen
  \item Weginformationen
\end{itemize}

\paragraph{Pfadinformationen}
\label{sec:pfadinformationen}

Wenn ein Pfad gefunden wurde, werden hier die Länge und die Passierzeit angezeigt\footnote{Siehe Abbildung \ref{fig:pfadinformationen}}.
Mit dem Button rechts oben können weitere Informationen eingeblendet werden.

\begin{figure}[htbp]
  \centering
  \includegraphics[width=0.4\textwidth]{Bilder/Pfadinformationen.png}
  \caption{Pfadinformationen}
  \label{fig:pfadinformationen}
\end{figure}

\paragraph{Knoteninformationen}
\label{sec:knoteninformationen}

Nach der Markierung eines Knotens werden in diesem Bereich die Knotendaten eingetragen.
Es handelt sich um die Id und die geographischen Daten, die direkt aus \ac{osm} übernommen werden\footnote{Siehe Abbildung \ref{fig:knoteninformationen}}.

\begin{figure}[htbp]
  \centering
  \includegraphics[width=0.4\textwidth]{Bilder/Knoteninformationen.png}
  \caption{Knoteninformationen}
  \label{fig:knoteninformationen}
\end{figure}

Die drei Buttons rechts oben im Knoteninformationenbereich\footnote{Siehe Abbildung \ref{fig:knotenbuttons}} führen verschiedene Aktionen aus.
Mit dem linken kann der aktuell selektierte Knoten als Start und mit dem mittleren als Ziel gesetzt werden.
Ein als Start oder Ziel gesetzter Knoten wird mit einem entsprechenden Marker versehen\footnote{Siehe Abbildungen \ref{fig:startknoten} und \ref{fig:zielknoten}}.
Mit dem rechten Button können weitere Informationen über die aktuell selektierte Knotenkonstellation eingeblendet werden.

\begin{figure}[htbp]
  \centering
  \includegraphics[width=0.15\textwidth]{Bilder/Knotenbuttons.png}
  \caption{Knotenbuttons}
  \label{fig:knotenbuttons}
\end{figure}

\begin{figure}[htbp]
  \centering
  \includegraphics[width=\textwidth]{Bilder/Startknoten.png}
  \caption{Startknoten}
  \label{fig:startknoten}
\end{figure}

\begin{figure}[htbp]
  \centering
  \includegraphics[width=\textwidth]{Bilder/Zielknoten.png}
  \caption{Zielknoten}
  \label{fig:zielknoten}
\end{figure}

\paragraph{Weginformationen}
\label{sec:weginformationen}

Hier werden die Daten eines selektierten Weges angezeigt.
Neben einer Weg-Id werden hier die Länge, die aktuell maximale Geschwindigkeit des Weges, die Passierzeit und die Knoten angezeigt, aus denen der Weg besteht\footnote{Siehe Abbildung \ref{fig:weginformationen}}.
Die Knoten können durch einen Klick auf ``Nodes'' eingeblendet werden.

Unter der maximalen Geschwindigkeit sind drei Buttons zu sehen\footnote{Siehe Abbildung \ref{fig:wegeditierbuttons}}.
Mit diesen Buttons kann die maximale Geschwindigkeit editiert werden.
Der linke Button schaltet in den Editiermodus.
Im Editiermodus ist die maximale Geschwindigkeit editierbar und die beiden rechten Buttons sind aktiv.
Anschließend kann die neue Geschwindigkeit eingegeben werden.
Mit dem mittleren Button kann diese gespeichert werden.
Mit dem rechten Button wird der Editiermodus verlassen, ohne die Änderung zu übernehmen.

\begin{figure}[htbp]
  \centering
  \includegraphics[width=0.3\textwidth]{Bilder/Weginformationen.png}
  \caption{Weginformationen}
  \label{fig:weginformationen}
\end{figure}

\begin{figure}[htbp]
  \centering
  \includegraphics[width=0.15\textwidth]{Bilder/Wegeditierbuttons.png}
  \caption{Wegeditierbuttons}
  \label{fig:wegeditierbuttons}
\end{figure}

\subsubsection{Monitoring}
\label{sec:monitoring}

In der Monitoring-Ansicht kann der aktuelle Zustand des Systems abgelesen werden.
Diese ist in drei Bereiche unterteilt\footnote{Siehe Abbildung \ref{fig:monitoring}}:

\begin{itemize}
  \item Ameisen-Monitoring (Abbildung \ref{fig:ameisen-monitoring})
  \item Knoten-Monitoring (Abbildung \ref{fig:knoten-monitoring})
  \item Zeitpunkt der letzten Aktualisierung (Abbildung \ref{fig:letztes-update})
\end{itemize}

\begin{figure}[htbp]
  \centering
  \includegraphics[width=\textwidth]{Bilder/Monitoring.png}
  \caption{Monitoring}
  \label{fig:monitoring}
\end{figure}

\subsubsection{Ameisen-Monitoring}
\label{sec:ameisen-monitoring}

In diesem Bereich ist oben eine Tabelle zu sehen.
In dieser ist jeweils die Gesamtanzahl der erzeugten, der das Ziel erreichten, der in einer Sackgasse angekommenen und der das maximale Alter überschrittenen Ameisen aufgeführt.
Unter der Tabelle ist das durchschnittliche Ameisenalter, die Leerlaufzeit der Ameisen und die Zeit, die zum Auswählen des nächsten Knotens benötigt wurde, angezeigt.

\begin{figure}[htbp]
  \centering
  \includegraphics[width=0.7\textwidth]{Bilder/Ameisen-Monitoring.png}
  \caption{Ameisen-Monitoring}
  \label{fig:ameisen-monitoring}
\end{figure}

\subsubsection{Knoten-Monitoring}
\label{sec:knoten-monitoring}

Hier sind folgende Angaben zu finden:

\begin{itemize}
  \item Durchschnittliche Dauer, die zum Erzeugen von Ameisen benötigt wird.
  \item Durchschnittliche Anzahl der Ameisen, die ein Knoten pro Sekunde verarbeitet.
  \item Durchschnittliche Dauer, die für die Verarbeitung einer Ameise nötig ist.
  \item Durchschnittliche Dauer, die für die Aktualisierung der Datenstrukturen der Knoten gebraucht wird.
\end{itemize}

\begin{figure}[htbp]
  \centering
  \includegraphics[width=0.7\textwidth]{Bilder/Knoten-Monitoring.png}
  \caption{Knoten-Monitoring}
  \label{fig:knoten-monitoring}
\end{figure}

\begin{figure}[htbp]
  \centering
  \includegraphics[width=0.3\textwidth]{Bilder/Letztes-Update.png}
  \caption{Letztes Update}
  \label{fig:letztes-update}
\end{figure}

\subsection{Routing}
\label{sec:routing}

Wenn sowohl Start als auch Ziel wie in Abschnitt \ref{sec:knoteninformationen} beschrieben gesetzt wurden, sucht AntScout eine Route vom Start zum Ziel und zeigt diese an.
Eine Beispielroute ist in Abbildung \ref{fig:route} zu sehen.
Jetzt kann ein beliebiger Weg, unabhängig davon, ob dieser auf der Route liegt, selektiert und dessen maximale Geschwindigkeit wie in Abschnitt \ref{sec:weginformationen} beschrieben verändert werden.
Eine starke Verringerung der Geschwindigkeit veranlasst das System nach einer schnelleren Route zu suchen und diese anzuzeigen.
Eine alternative Route wird in Abbildung \ref{fig:alternative-route} gezeigt.

\begin{figure}[htbp]
  \centering
  \includegraphics[width=\textwidth]{Bilder/Route.png}
  \caption{Route}
  \label{fig:route}
\end{figure}

\begin{figure}[htbp]
  \centering
  \includegraphics[width=\textwidth]{Bilder/Alternative-Route.png}
  \caption{Alternative Route}
  \label{fig:alternative-route}
\end{figure}

\subsection{Konfiguration}
\label{sec:konfiguration}

\textbf{Achtung!}
Hier sollten nur fortgeschrittene Benutzer etwas ändern.
Es besteht die Gefahr einer ungültigen Konfiguration.
Dann gibt AntScout beim Starten eine Fehlermeldung aus und startet nicht mehr richtig.

Im Verzeichnis \texttt{AntScout/src/main/resources} befindet sich die Datei \texttt{reference.conf}.
Dort sind die Einstellungen für AntScout zu finden.
Die Datei ist eine Textdatei und kann mit einem Texteditor bearbeitet werden.

Die Konfiguration ist in mehrere Ebenen unterteilt, wobei \texttt{ant-scout} die oberste Ebene darstellt.
Jede ebene ist in geschweifte Klammern eingeschlossen.

Jeder Parameter ist durch einen Namen und einen Wert definiert.

\begin{lstlisting}
name = wert
\end{lstlisting}

Jede Änderung der Konfiguration wird erst nach einem Neustart der Applikation wirksam.

\begin{description}
  \item[ant-net] In diesem Abschnitt können die Parameter angepasst werden, die die Arbeitsweise von AntNet - des im Hintergrund arbeitenden Routingalgorithmus - beeinflussen.
    Diese Parameter werden in den nächsten Abschnitten nur grob beschrieben.
    Detailliertere Ausführungen sind in der entsprechenden Literatur (\cite{DorigoStuetzle2004} oder \cite{Walther2006}) zu finden.
    \begin{description}
      \item[a] Parameter \texttt{a} für die Squash-Funktion.
      \item[alpha] Relatives Gewicht der heuristischen Information, die in die Berechnung der Wahrscheinlichkeiten einfließt.
      \item[ants-launch] Parameter, die zum Erzeugen der Ameisen benötigt werden.
        \begin{description}
          \item[distance] Die von einem Knoten aus erreichbaren Ziele werden in Gruppen unterteilt.
            Dieser Parameter (Angabe in Metern) entscheidet, in welchen Abständen eine neue Gruppe erzeugt wird.
          \item[interval] Intervall in Millisekunden, in dem Ameisen mit Zielen aus der am weitesten entfernten Gruppe erzeugt werden.
          \item[interval-increment] Mit abnehmender Entfernung wird pro Gruppe dieser Wert (Angabe in Millisekunden) zum \texttt{interval} hinzu addiert.
        \end{description}
      \item[max-ant-age] Maximales Alter einer Ameise in Millisekunden.
        Wenn die Ameise ihr Ziel nicht innerhalb dieser Zeit erreicht hat, wird sie aus dem System entfernt.
      \item[best-way-pheromone] Mit diesem Wert wird der beste Weg in der Präpheromonmatrix initialisiert, der als nächstes auf dem Weg zum Ziel besucht werden sollte.
      \item[c1] Gewichtungsfaktor, der den Einfluss des Verhältnisses der besten Fahrzeit zur aktuellen Fahrzeit bei der Berechnung der Verstärkung angibt.
      \item[c2] Gewichtungsfaktor, der den Einfluss der Vertrauenswürdigkeit der aktuellen Fahrzeit bei der Berechnung der Verstärkung angibt.
      \item[rho] Faktor, der die Anzahl der Messungen bestimmt, die zum Berechnen des Mittelwertes und der Varianz des lokalen statistischen Modells verwendet werden.
      \item[w-max] Größe des gleitendes Beobachtungsfensters des lokalen statistischen Modells.
        Sollte nach \cite{DorigoStuetzle2004} wie folgt berechnet werden: $w-max = 5 \frac{c}{rho}$ mit $c <= 1$.
      \item[z] Parameter \texttt{z} für die Berechnung der Verstärkung.
    \end{description}
  \item[default-speeds] Standardgeschwindigkeiten für verschiedene \ac{osm}-Wegklassen.
    Diese Geschwindigkeiten werden den Wegen zugewiesen, wenn in der \ac{osm}-Wegdefinition keine maximale Geschwindigkeit definiert ist.
  \item[epsilon] Schwellwert für den Vergleich von zwei Double-Werten.
  \item[map] Dieser Parameter steuert, welche Karte verwendet werden soll.
    Es sind mehrere Karten in verschiedenen Größen vordefiniert.
    Die Zeile, die die aktuell verwendete Karte enthält, ist die einzige, die nicht mit einem \#-Zeichen beginnt.
    Dieser Zeile muss ein \#-Zeichen vorangestellt werden, damit sie nicht mehr verwendet wird.
    Vor der Zeile, die die gewünschte Karte enthält, muss das \#-Zeichen entfernt werden.
    \begin{lstlisting}[caption = Karten, label = lst:karten]
# Karte, die verwendet werden soll.
# 104 Knoten, 99 Quellen und 100 Ziele
# map = maps/Bahrenfeld-Gross-Flottbek-Othmarschen-Ottensen.osm
# 85 Knoten, 83 Quellen und 83 Ziele
# map = maps/Blankenese-Wedel.osm
# 47 Knoten, 43 Quellen und 45 Ziele
map = maps/Altona-50-Knoten.osm
# 14 Knoten, 12 Quellen und Ziele
# map = maps/Altona-Kreis.osm
# 142 Knoten, 138 Quellen und Ziele
# map = maps/Altona-Wedel.osm
# 57 Knoten, 57 Quellen und 56 Ziele
# map = maps/Wedel.osm
    \end{lstlisting}
  \item[max-path-length] Die Suche nach einem Pfad wird abgebrochen, wenn der Pfad diese Länge erreicht.
  \item[monitoring] Einstellungen für das Monitoring.
    \begin{description}
      \item[buffer-size] Begrenzung der Buffer-Größen, in denen die Monitoring-Daten gespeichert werden.
        Der Wert 0 schaltet die Buffer-Begrenzung aus.
      \item[data-processing-interval] Intervall in Sekunden, in dem Monitoring-Daten aufbereitet und im Front-End angezeigt werden.
        Bei 0 werden die Monitoring-Daten nicht aufbereitet und auch nicht im Front-End angezeigt.
      \item[enabled] Schaltet das Monitoring ein oder aus.
        Monitoring senkt die Performance und sollte nur wenn nötig eingeschaltet werden!
    \end{description}
  \item[relevant-highways] \ac{osm}-Wegklassen, die für den AntNet-Algorithmus berücksichtigt werden sollen.
  \item[trace-is-enabled] Flag, ob detaillierte (Log-)Ausgaben erzeugt werden sollen.
    Senkt die Performance und sollte nur zur Fehlersuche eingeschaltet werden!
\end{description}

% \section{Fehlermeldungen}
% \label{sec:fehlermeldungen}

\section{Wiederanlaufbedingungen}
\label{sec:wiederanlaufbedingungen}

Sollte die Anwendung einmal abstürzen oder/und nicht mehr reagieren, kann sie wie in Abschnitt \ref{sec:programmende} beschrieben beendet werden.
In Abschnitt \ref{sec:start} ist geschildert, wie die Anwendung erneut gestartet werden kann.


\chapter{Entwicklerhandbuch}
\label{chap:entwicklerhandbuch}

\section{Entwicklungskonfiguration}
\label{sec:entwicklungskonfiguration}

\subsection{Hardware}
\label{sec:hardware}

\begin{description}
  \item[Prozessor] Intel Core i5-2520M, 4 $\times$ 2.50 GHz
  \item[Speicher] 7,6 GiB
  \item[Grafikkarte] Intel Sandybridge Mobile
\end{description}

\subsection{Software}
\label{sec:software}

\begin{description}
  \item[Betriebssystem] Ubuntu 12.10, 64 Bit
  \item[\ac{ide}] IntelliJ IDEA 11.1.4
\end{description}

\subsubsection{Programmiersprachen}
\label{sec:programmiersprachen}

\begin{description}
  \item[Scala\footnotemark] \footnotetext{\url{http://www.scala-lang.org/}} wird im Back-End als Programmiersprache eingesetzt, da diese dank dem Aktorenkonzept die Entwicklung von nebenläufigen Anwendungen vereinfacht.
    Scala läuft auf der \ac{jvm} und ist damit plattformunabhängig.
  \item[CoffeeScript\footnotemark] \footnotetext{\url{http://coffeescript.org/}} kommt im Front-End zum Einsatz.
    Diese JavaScript ähnliche Sprache transkompiliert in JavaScript.
    CoffeeScript verbessert die Lesbarkeit von JavaScript und führt zusätzliche Funkionalität wie ``List Comprehension'' oder ``Pattern Matching'' ein.
\end{description}

\subsubsection{Bibliotheken}
\label{sec:bibliotheken}

\begin{description}
  \item[Config\footnotemark] \footnotetext{\url{https://github.com/typesafehub/config}} ermöglicht es, eine textbasierte Konfiguration für eine Anwendung zu erstellen.
    Es werden verschiedene Formate unterstützt.
    Zusätzlich stehen verschiedene Methoden zur Verfügung, um den Inhalt der Konfiguration auszulesen.
  \item[Lift\footnotemark] \footnotetext{\url{http://liftweb.net/}} ist ein mächtiges und sicheres Web-Framework.
    Damit ist es in Scala möglich Webanwendungen zu erstellen.
  \item[jQuery\footnotemark] \footnotetext{\url{http://jquery.com/}} ist eine JavaScript-Bibliothek und vereinfacht in JavaScript den Umgang mit HTML, Events oder Ajax.
  \item[Leaflet\footnotemark] \footnotetext{\url{http://leafletjs.com/}} ist eine kleine und moderne Bibliothek, die für die Darstellung interaktiver Karten genutzt wird.
  \item[RequireJS\footnotemark] \footnotetext{\url{http://requirejs.org/}} dient der Modularisierung von JavaScript.
  \item[Underscore\footnotemark] \footnotetext{\url{http://underscorejs.org/}} erweitert JavaScript um viele nützliche Funktionen.
\end{description}

\subsubsection{Erstellung}
\label{sec:erstellung}

Die Erstellung wird mit \ac{sbt}\footnote{\url{http://www.scala-sbt.org/}} realisiert, einem Standardwerkzeug, das in der Scala-Welt für diesen Zweck genutzt wird.

\section{Design und Implementierung}
\label{sec:design-und-implementierung}

In den folgenden Abschnitten wird die Implementierung der ersten Teilaufgabe von AntScout beschrieben und die zugrunde liegenden Designentscheidungen diskutiert.
Es wird nur grob auf die Abläufe und Verfahren eingegangen.
Detaillierte Ausführungen können dem gut dokumentierten Sourcecode entnommen werden.

\subsection{Architektur}
\label{sec:architektur}

In der Abbildung \ref{fig:architektur} ist die Architektur von AntScout abgebildet.

\begin{figure}[htbp]
  \centering
  \includegraphics[width=\textwidth]{Bilder/Architektur.pdf}
  \caption{Architektur}
  \label{fig:architektur}
\end{figure}

\subsection{Kartenabruf}
\label{sec:kartenabruf}

\ac{osm} stellt ein \ac{rest}-\ac{api} zum Kartenabruf im \ac{xml}-Format zur Verfügung.
Es wird allerdings eine andere Möglichkeit genutzt, um die \ac{osm}- und auch lokale Ressourcen zu schonen.
Regelmäßig aktualisierte \ac{osm}-Karten nach Regionen unterteilt werden auch kostenlos von verschiedenen Dienstleistern angeboten.
Es wird die Hamburg-Karte vom Dienstleister Geofabrik heruntergeladen und vorverarbeitet.
Für die Vorverarbeitung von \ac{osm}-Karten wird die Anwendung Osmosis\footnote{\url{http://wiki.openstreetmap.org/wiki/Osmosis}} genutzt.
Die Karte wird im ersten Schritt gefiltert, sodass nur relevante Wege überbleiben und Knoten, die zu diesen Wegen gehören.
Im zweiten Schritt werden aus der ursprünglichen Karte rechteckige Ausschnitte extrahiert.
Diese Ausschnitte werden anschließend in AntScout geladen und weiterverarbeitet.

\ac{sbt} bietet die Möglichkeit, beliebige Aufgaben als sogenannte Tasks zu implementieren.
Zwischen diesen und bereits existierenden Tasks können Abhängigkeiten definiert werden.
Damit die Vorverarbeitungsschritte nicht jedes mal manuell ausgeführt werden müssen, wurden diese als \ac{sbt}-Tasks umgesetzt und werden bei Bedarf vor dem Start der Anwendung ausgeführt.
Die ausführlich dokumentierte Umsetzung ist in der Datei \texttt{project/Build.scala} zu finden.

\subsection{Kartenimport}
\label{sec:kartenimport}

Nach dem Start der Anwendung wird die in der Konfiguration definierte Karte geladen.
Die Karte wird durch einen globalen Singleton \texttt{OsmMap} repräsentiert, damit auf die Daten aus jeder anderen Klasse zugegriffen werden kann.
Zuerst werden die Knoten geladen.
Aus jedem Knoten wird eine \texttt{OsmNode}-Instanz erzeugt und in \texttt{OsmMap} gespeichert.
Im nächsten Schritt werden die Wege geladen.
Hier werden während des Ladevorganges verschiedene Wegeigenschaften - Tags genannt - wie Wegklasse, Name oder maximale Geschwindigkeit analysiert.
Die Bestimmung der maximalen Geschwindigkeit ist in drei Stufen realisiert:

\begin{enumerate}
  \item Es wird angenommen, dass die maximale Geschwindigkeit im \texttt{maxspeed}-Tag definiert ist.
    Es wird versucht den Wert dieses Tags auszulesen.
    Im Falle einer gültigen Zahl wird angenommen, dass der Wert in km/h angegeben ist und in m/s umgewandelt.
  \item Ist das \texttt{maxspeed}-Tag nicht vorhanden oder der Wert keine gültige Zahl, wird versucht die maximale Geschwindigkeit anhand der Wegklasse zu bestimmen.
    Diese ist im \texttt{highway}-Tag definiert.
    In der Konfiguration ist für jede Wegklasse eine Standardgeschwindigkeit abgelegt.
    Wenn die definierte Wegklasse mit einer der Wegklassen aus der Konfiguration übereinstimmt, wird diese Geschwindigkeit verwendet.
  \item Wenn die beiden ersten Schritte nicht erfolgreich sind, wird im letzten Schritt eine in der Konfiguration definierte Standardgeschwindigkeit verwendet.
\end{enumerate}

Im letzten Schritt des Wegladevorganges wird überprüft, ob der Weg eine Einbahnstraße repräsentiert.
Je nach Ergebnis wird eine \texttt{OsmOneWay}- oder \texttt{OsmWay}-Instanz erzeugt und in \texttt{OsmMap} gespeichert.

Im letzten Schritt des Kartenladevorganges wird eine Adjazenzmatrix ähnliche Datenstruktur berechnet.
Es handelt sich um eine Abbildung von Knoten auf Wege, die diesen Knoten als Bestandteil haben.

\subsection{Kartenumwandlung}
\label{sec:kartenumwandlung}

Die Umwandlung der \ac{osm}-Karte in einen AntNet-Graphen, der auch durch einen globalen Singleton \texttt{AntMap} repräsentiert wird, geschieht in mehreren Schritten.

\subsubsection{Filterung}
\label{sec:filterung}

Die Anzahl der in AntNet erzeugten Ameisen hängt von der Größe der Karte bzw. der Anzahl der Knoten ab.
Um die Anzahl der Ameisen und damit auch die Computerlast, auf dem die Anwendung läuft, zu reduzieren, werden die nicht relevanten Wege im ersten Schritt herausgefiltert.
Die relevanten Wege sind in der Konfiguration definiert:

\begin{lstlisting}
relevant-highways = [motorway,motorway_link,trunk,trunk_link,primary,primary_link,secondary,secondary_link,tertiary]
\end{lstlisting}

\subsubsection{Reduzierung}
\label{sec:reduzierung}

In AntNet werden in regelmäßigen Zeitintervallen an jedem Knoten Ameisen erzeugt.
Im zweiten Schritt wird die Anzahl der Knoten auf ein Minimum reduziert.
Für AntNet sind nur Knoten relevant, die Straßenenden und Kreuzungen repräsentieren.
Das sind alle Knoten vom Grad $= 1$ und Grad $\geq 3$.
Alle Straßen(-segmente), die durch Knoten vom Grad $= 2$ verbunden sind, werden zu einem \texttt{AntWay} oder \texttt{AntOneWay}, wenn es sich um einen Weg handelt, der eine Einbahnstraße repräsentiert, zusammengefasst und in \texttt{AntMap} gespeichert.

\subsubsection{Knotenberechnung}
\label{sec:knotenberechnung}

Nach der Zusammenfassung der Wege, werden die endgültigen Knoten für AntNet berechnet.
Die relevanten Knoten setzen sich aus Start- und Endknoten der zusammengefassten Wege zusammen.

\subsubsection{Aus- und eingehende Wege}
\label{sec:aus-und-eingehende-wege}

Eine Straßenkarte kann nur durch einen gerichteten Graphen repräsentiert werden.
In diesem letzten Schritt werden zwei Abbildungen berechnet.
Eine repräsentiert die ausgehenden und die andere die eingehenden Wege pro Knoten.

\subsection{\acs*{rest}}
\label{sec:rest}

Das Back-End bietet ein \ac{rest}-\ac{api} an, um z.B. die Knoten oder die Wege abzufragen.
In der Tabelle \ref{tab:rest-api} ist eine Übersicht der zur Verfügung stehenden Methoden abgebildet.
Sowohl die Parameter an das \ac{api} als auch deren Antwort sind im \ac{json}-Format.

\begin{table}[htbp]
  \centering
  \begin{tabularx}{\textwidth}{lllX}
    Ressource & Methode & Parameter & Antwort \\
    \toprule
    \texttt{node/\{node-id\}} & GET & - & Aus- und eingehende Wege des Knoten mit der Id \{node-id\}. \\
    \midrule
    \texttt{nodes} & GET & - & Alle \texttt{AntMap}-Knoten. \\
    \midrule
    \texttt{ways} & GET & - & Alle \texttt{AntMap}-Wege. \\
    \midrule
    \texttt{way/\{way-id\}} & GET & - & Daten des \texttt{AntMap}-Weges mit der Id \{way-id\}. \\
    \midrule
    \texttt{way/\{way-id\}} & PUT & \texttt{maxSpeed: \{maxSpeed\}} & Daten des veränderten \texttt{AntMap}-Weges mit der Id \{way-id\}. \\
    \bottomrule
  \end{tabularx}
  \caption{\acs*{rest}-\acs*{api}}
  \label{tab:rest-api}
\end{table}

\section{Grundlegende Datenstrukturen}
\label{sec:grundlegende-datenstrukturen}

\subsection{OsmMap}
\label{sec:osm-map}

In diesem Singleton sind die Knoten und Wege gespeichert, die direkt aus den \ac{osm}-Daten geladen werden.
Zusätzlich ist hier die Abbildung von Knoten auf die adjazenten Wege zu finden.

\subsection{AntMap}
\label{sec:ant-map}

Diese Datenstruktur ist auch ein Singleton.
Hier sind die Knoten und Wege zu finden, die für den AntNet-Algorithmus relevant sind.
Diese Daten werden genutzt, um die \ac{rest}-Anfragen zu beantworten.

\section{Tests}
\label{sec:tests}

Während der Entwicklung wurde die Anwendung immer wieder gestartet und geprüft, ob sie sich erwartungsgemäß verhält.
Zusätzlich zum normalen Ablauf wurden immer wieder gewisse (Fehler-)Situationen provoziert und das Verhalten getestet.


\bibliography{Literatur}

\end{document}
